\chapter*{Conclusions}
Measurements of the Higgs boson differential fiducial cross sections are performed, based on the full proton-proton collisions dataset, recorded at center-of-mass energy of $\sqrt{s} = 13$ TeV by the ATLAS experiment and corresponding to the full Run2 dataset with an integrated luminosity of $139$ fb$^{-1}$.
\\\\
Measurements are performed for a wide set of observables, sensitive to the Higgs boson and the jets kinematics for the selected events: the transverse momentum $p_T^{\gamma\gamma}$ and the rapidity $|y|_{\gamma\gamma}$ of the Higgs boson measured in the di-photon system, sensitive to the bottom and charm quark Yukawa couplings of Higgs boson \cite{Bishara_2017} and to new heavy particles coupling to the Higgs boson for the $p_T^{\gamma\gamma}$ distribution and responsive to the modeling of the production mechanism and to the parton distribution functions (PDF) of the colliding protons for the $|y|_{\gamma\gamma}$ distribution respectively, plus several variables related the jets, such as the multiplicity of jets $N_{jets}^{30}$ and the transverse momentum $p_T^{j1, 30}$ of the leading jet associated with the Higgs boson production, the invariant mass $m_{jj}^{30}$ and the azimuthal angular difference $\Delta\phi_{jj}^{30}$ of the two leading jets, where the $p_T^{j1}$ distribution is sensitive  to the relative contributions of the different Higgs production mechanism, while the $m_{jj}$ distribution has sensitivity to the Vector Boson Fusion (VBF) production mechanism. Finally, the angular variable $\Delta \phi_{jj}$ is sensitive to the spin and CP quantum numbers of the Higgs boson.
\\
The fiducial selection region constists of a Higgs boson decaying into two isolated photons of transverse momentum greater than 35\% and 25\% of the diphoton invariant mass for the leading and sub-leading photon respectively, and with $|\eta| < 2.37$, excluding the the barrel-endcap transition region of $1.37 < |\eta| < 1.52$.
\\\\
The comparison between the observed values for the cross sections measured from data and the values predicted from the Standard Model shows a good overall agreement, not exhibiting any deviation from the Standard Model expectation.
\\
The uncertainties decomposition is dominated by the statistical component, while for the systematic component the energy resolution and the spurious signal are the biggest uncertainty sources.
\\\\
The background modelling is probably the most complicated part of the analysis.
The spurious signal analysis shows that in many categories the selected background function doesn't pass the selection conditions and it is chosen to be the one with the minimum spurious signal value. To improve this, a smoothing of the background templates, in order to lower statystical fluctuations may be considered in the future.
\\\\
Finally, a luminosity extrapolation has been performed, with the High-Luminosity LHC luminosity.