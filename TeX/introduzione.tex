\chapter*{Introduction}
The Higgs boson is a fundamental scalar particle whose existence is predicted by the Higgs mechanism proposed in 1964 by Higgs, Brout, Englert, Guralnik, Hagen and Kibble, responsible for the spontaneous symmetry breaking. The introduction of this mechanism in the Standard Model, the quantum field theory which best describes the behaviour of elementary particles, leads naturally to the appearance of mass terms for all massive elementary particles, without explicitly breaking the fundamental symmetry of the underlying theory. On July 4th, 2012, the discovery of a new particle with a mass of about $125$ GeV and compatible with the Higgs boson was announced by the ATLAS and CMS collaborations. Since then, new and more precise measurements were performed by the two experiments, improving the knowledge of the particle's properties, such as mass, couplings, spin and parity.
\\\\
The analysis presented in this work is based on the full proton-proton collision data set recorded at center-of-mass energy of 13 TeV by the ATLAS experiment (the so-called Run2, amounting to an integrated luminosity of 139 fb$^{-1}$) at the Large Hadron Collider (LHC) during the period 2015-2018. The ATLAS experiment is particularly involved in the study of the Higgs boson particle.
\\\\
This thesis is about the measurement of the differential fiducial cross sections of Higgs boson production, for several observables, based on the di-photon $H \rightarrow \gamma\gamma$ decay channel.
This channel, although penalized by a low Branching Ratio ($2.27 \times 10^{-3}$) compared to other decay channels, has the advantages of a good efficiency, a fully reconstructed final state, and a very good invariant mass resolution of the Higgs peak. In this thesis, the full analysis is described with some original additions and variations with respect to the standard ATLAS published papers.
\\\\
The analysis I carried out consists of several steps to measure the observed fiducial differential cross sections, compared to the predicted values of the SM. The analysis is based on signal + background fits of the $m_{\gamma\gamma}$ distribution for different categories, each one to measure the cross section in a particular fiducial region. As a starting point, I developed a code using MonteCarlo simulations, in order to choose from a range of selected functional forms the one which best describes the background spectra. Similarly, employing a simulated set of signal events, I tuned a model for the Higgs boson signal $m_{\gamma\gamma}$ shape. Using both backgrounds and signal parametrisations, I evaluated the potential mismodelling for all background functional forms. In order to correct for detector effects, the reconstructed quantities are unfolded to particle level using two methods: the bin-by-bin unfolding method and the matrix unfolding method. Moreover, I studied the distortions and the biases due to the experimental effects through the introduction of several experimental systematic uncertainties.
\\\\
With all the ingredients explained above, I set up a statistical model describing the $m_{\gamma\gamma}$ distribution in each bin of the variable of interest. The cross sections and their uncertainties are extracted with a maximum likelihoood fit on data, finding in this way the observed cross section values for the Higgs boson distributions by a fitting procedure. The different and original approach I contributed in this part with respect to the previous ATLAS papers constists in treating the Dalitz $H \rightarrow \gamma\gamma *$ channel and the $H \rightarrow \gamma\gamma$ events coming from outside the fiducial phase space as separate and distinct categories, in order to simplify the matrix unfolding method. In the previous ATLAS analyses the procedure was split in two steps: first,  the signal extraction from the $m_{\gamma\gamma}$ fit and then the unfolding. In this work I implemented both steps together in the likelihood optimization. Finally, I compared the observed cross sections derived in this way to the cross sections predicted by the Standard Model. The results agree with the Standard Model predictions and with the previous ATLAS published results.
\\
Moreover, I've performed an extrapolation to a higher integrated luminosity, in order to study the future High-Luminosity LHC (HL-LHC) performance and how the larger statistics will affect the total error.
\\\\
The manuscript is articulated in five chapters, each of them focusing on a theoretical or experimental aspect regarding the Higgs boson and the ATLAS detector. In Chapter \ref{capitolo_1} I describe the general features of the Standard Model of Particle Physics from a theoretical point of view, putting the attention in the last part to the Higgs sector and to its experimental hilights. In Chapter \ref{capitolo_2} and Chapter \ref{capitolo_3} I illustrate a technical framework of the LHC accelerator and the ATLAS detector, focusing on the calorimetric system, which plays a crucial role in the detection of the Higgs' di-photon decay channel studied in this thesis. In Capther \ref{capitolo_4} I discuss how particles are reconstructed and identified by the ATLAS detector. The Chapter \ref{capitolo_5} is the main body of the manuscript and describes in details the analysis I set up in order to obtain the $H \rightarrow \gamma\gamma$ differential cross sections. As the last contribute, I report the conclusions of my analysis with some preliminary study for the High-Luminosity LHC configuration.
