\chapter*{Acknowledgments}
In un anno di cose ne succedono molte. In un anno tutto cambia. In un anno ci si può sentire un po' più bravi in qualcosa che fino a poco tempo prima si guardava solamente da lontano e con un po' di celata ammirazione. Un anno dopo si scrive l'ultima pagina di un lavoro che ha richiesto fatica e impegno, ma anche tanta soddisfazione. Dopo un anno questa tesi è finalmente completa.
\\
I ringraziamenti, si sà, rischiano di essere la parte più difficile di un lavoro ai miei occhi già complesso di per sè, ma che non può esimersi nel trovare al suo interno un posto dedicato a tutte quelle persone che, ognuna a proprio modo e con i propri mezzi, hanno contribuito alla realizzazione di questa tesi.
\\
Inizierò quindi così questo mio tentativo di tenere vicine tutte quelle persone che avrebbero meritato un sentito abbraccio, ma che purtroppo questo strano periodo ci costringe a rimandare.
\\\\
Desidero innanzitutto ringraziare Marcello Fanti, per avermi condotto in questo mondo fatto di particelle microscopiche e di macchine assurde e per avermi fatto innamorare di tutto ciò. A lui và la mia più sincera stima per essermi stato da guida in questo lungo cammino. Desidero ringraziare Ruggero Turra per tutti gli insegnamenti che è stato in grado di darmi e per la pazienza e la costanza che ha avuto nonostante i mille errori che questo processo ha comportato. Senza di lui probabilmente questa tesi non avrebbe mai visto la luce.
\\
Poichè ricerca e vita si mescolano e spesso si confondono, in questo lavoro di tesi non posso prescindere dal ringraziare coloro che nella mia quotidianità mi sono stati accanto. Desidero ringraziare la mia famiglia che, nonostante le fatiche di questo ultimo periodo, ha saputo incoraggiarmi sempre e comunque trovando le parole giuste al momento giusto. Desidero infine ringraziare in un modo speciale Gaia, compagna di mille avventure, per esserci stata in ogni occasione e per avermi saputo far scendere da quell'albero e toccare così finalmente terra perchè si sà, in due tutto diventa più facile. 
\\\\
Dopo vent'anni, questo è finalmente il punto di arrivo e queste sono davvero le ultime righe. Che l'ultimo giro di giostra abbia inizio.